\section{Page Replacement Policies}

\paragraph{Page Replacement --- Naive}
\begin{itemize}
  \item \textbf{step 1: save/clear victim page}:
  \begin{itemize}
    \item drop page if fetched from disk and clean
    \item \emph{dirty}: write back modifications if from disk and dirty (unless \code{MAP_COPY})
    \item \emph{non-dirty}: write page file/swap partition otherwise (e.g., stack, heap memory)
  \end{itemize}
  \item \textbf{step 2: unmap page from old AS}: invalidate PTE, flush cache
  \item \textbf{step 3: prepare new page}: null page or load new contents
  \item \textbf{step 4: map page frame into new AS}: invalidate PTE, flush cache
\end{itemize}

\paragraph{Page Replacement --- Buffering}
\begin{itemize}
  \item \textbf{Problem}: naive page replacement encompasses two I/O transfers
  \begin{itemize}
    \item[$ \to $] both operations block page fault from completing
  \end{itemize}
  \item \textbf{Goal}: reduce I/O from critical page fault path to speed up page faults
  \item \textbf{Idea}: keep pool of free page frames (\emph{pre-cleaning}):
  \begin{itemize}
    \item \emph{on page fault}: use page frame from free pool
    \item \emph{cleaning}: daemon cleans, reclaims and scrubs pages for free pool in background
    \item[$ \to $] smooths out I/O, speeds up paging significantly
  \end{itemize}
  \item \textbf{Remaining problem}: which pages to select as victims?
  \begin{itemize}
    \item \emph{goal}: identify page that has left working set of its processes, add to free pool
    \item \emph{success metric}: low overall page fault rate
  \end{itemize}
\end{itemize}

\paragraph{Page Replacement --- FIFO}
\begin{itemize}
  \item \textbf{Idea}: evict oldest fetched page in system
  \item \textbf{Belady's Anomaly}: using FIFO, for every number $ n $ of page frames you can construct a reference string that performs worse with $ n+1 $ frames
  \begin{itemize}
    \item[$ \to $] with FIFO it is possible to get more page faults with more page frames!
  \end{itemize}
\end{itemize}

\paragraph{Page Replacement --- oracle}
\begin{itemize}
  \item[=] optimal replacement strategy: replace page with next reference furthest in future
  \item \textbf{Problem}: future unpredictable
  \item \textbf{However}: good metric to check how well other algorithms perform
\end{itemize}

\paragraph{Page Replacement --- LRU}
\begin{itemize}
  \item \textbf{Goal}: approximate oracle page replacement
  \item \textbf{Idea}: past often predicts future well
  \item \textbf{Assumption}: page used furthest in past is used furthest in future
  \item \textbf{Cycle counter implementation}:
  \begin{itemize}
    \item have MMU write CPU's time stamp counter to PTE on every access
    \item \emph{page fault}: scan all PTEs to find oldest counter value
    \item \emph{advantage}: cheap at access if done in HW
    \item \emph{disadvantage}: memory traffic for scanning
  \end{itemize}
  \item \textbf{Stack implementation}:
  \begin{itemize} 
    \item keep doubly linked list of all page frames
    \item move each referenced page to tail of list
    \item \emph{advantage}: can find replacement victim in $ O(1) $
    \item \emph{disadvantage}: need to change 6 pointers at every access
  \end{itemize}
  \item $ \leadsto $ \textbf{No silver bullet}:
  \begin{itemize}
    \item \emph{observation}: predicting future based on past is not precise
    \item \emph{conclusion}: relax requirements --- maybe perfect LRU isn't needed? $ \Rightarrow $ approximate LRU
  \end{itemize}
\end{itemize}

\paragraph{LRU Approximation --- Clock page replacement}
\begin{itemize}
  \item aka \emph{second chance page replacement}
  \item \textbf{Precondition}: MMU sets reference bit in PTE
  \begin{itemize}
    \item supported natively by most hardware
    \item can easily emulate in systems with software managed TLB (e.g., MIPS)
  \end{itemize}
  \item \textbf{Store}: keep all pages in circular FIFO list
  \item \textbf{Searching} for victim: scan pages in FIFO's order
  \begin{itemize}
     \item if reference bit = 0 $ \to $ use page as victim and advance
     \item if reference bit = 1 $ \to $ set to 0, continue scanning
  \end{itemize}
  \item \textbf{Problem}: large memory $ \to $ most pages referenced before scanned
  \begin{itemize}
    \item \emph{solution}: use 2 arms, leading arm clears reference bit, trailing arm selects victim
  \end{itemize}
\end{itemize}

\paragraph{Replacement Strategies --- other}
\begin{itemize}
  \item \textbf{Random Eviction}: pick random victim
  \begin{itemize}
    \item dirt simple
    \item not overly horrible in reality
  \end{itemize}
  \item \textbf{Larger counter}: use $ n $-bit reference counter instead of reference bit
  \begin{itemize}
    \item \emph{least frequently used}: rarely used page not in a working set $ \to $ replace page with smallest count
    \item \emph{most frequently used}: page with smallest count probably just brought in $ \to $ replace page with largest count
    \item neither LFU nor MDU are common (no such hardware, not that great)
  \end{itemize}
\end{itemize}

\begin{summary}
  \begin{itemize}
    \item victim page frame needs to be selected by OS when handling page faults
    \begin{itemize}
      \item evicting page frame after page fault happens = not a good idea
      \item page buffering keeps eviction out of critical path
    \end{itemize}
    \item different victim selection policies exist
    \begin{itemize}
      \item FIFO $ \to $ Belady's Anomaly
      \item Oracle $ \to $ cannot predict the future
      \item Random $ \to $ unpredictable, never great but rarely very bad
      \item LRU $ \to $ hard to implement efficiently
    \end{itemize}
  \end{itemize}
\end{summary}