\section{Page Faults}

\paragraph{Page Faults --- Handling}
\begin{itemize}
  \item \textbf{Cause}: access to page currently not present in main memory
  \begin{itemize}
    \item[$ \to $] exception, invoking OS
  \end{itemize}
  \item \textbf{Process}:
  \begin{enumerate}
    \item OS checks access validity (requiring additional info)
    \item get empty frame
    \item load contents of requested page from disk into frame
    \item adapt page table
    \item set present bit of respective entry
    \item restart instruction causing page fault
  \end{enumerate}
\end{itemize}

\paragraph{Page Faults --- Latency}
\begin{itemize}
  \item \textbf{fault rate} $ 0 \leq p \leq 1 $
  \begin{itemize}
    \item $ p = 0 $: no page faults
    \item $ p = 1 $: every reference leads to page fault
  \end{itemize}
  \item \textbf{effective access time} (EAT):
  \begin{equation*}
    \text{EAT} = (1-p)*\text{memory access} + p*(\text{PF overhead} + \text{PF service time} + \text{restart overhead})
  \end{equation*}
\end{itemize}

\paragraph{Page Faults --- Performance Impact}
\begin{itemize}
  \item \textbf{memory access time}: 200ns
  \item \textbf{average page fault service time}: 8ms
  \item[$ \leadsto $] 1:1000 access-page-fault-rate $ \to $ $ \text{EAT} = 8.2\mu s $ $ \Rightarrow $ \emph{slowdown by factor 40}!
\end{itemize}

\paragraph{Page Faults --- Challenges}
\begin{itemize}
  \item \textbf{what to eject?}
  \begin{itemize}
    \item how to allocate frames among processes?
    \item which particular process's pages to keep in memory?
    \item see \emph{page frame allocation}
  \end{itemize}
  \item \textbf{what to fetch?}
  \begin{itemize}
    \item what if block size $ \neq $ page size?
    \item just one page needed? prefetch more?
  \end{itemize}
  \item \textbf{process resumption?}
  \begin{itemize}
    \item need to save state + resume
    \item process might have been in middle of instruction
  \end{itemize}
\end{itemize}

\paragraph{Page Faults --- What to fetch?}
\begin{itemize}
  \item bring in page causing fault
  \item \textbf{pre-fetch} sourrounding pages?
  \begin{itemize}
    \item reading two disk blocks is approximately as fast as reading one
    \item as long as there is no track/head switch, seek (disk) time dominates
    \item application exhibits spatial locality = big win
  \end{itemize}
  \item \textbf{pre-zero} pages?
  \begin{itemize}
    \item don't want to leak information between processes
    \item need 0-filled pages for stack, heap, .bss, \dots
    \item \emph{zero on demand}?
    \item keep pool of 0-pages filled in background when CPU is idle?
  \end{itemize}
\end{itemize}

\paragraph{Page Faults --- Process resumption}
\begin{itemize}
  \item hardware provides info about page fault
  \begin{itemize}
    \item (Intel: \code{\%cr2} contains faulting virtual address)
  \end{itemize}
  \item \textbf{Context}: OS needs to figure out fault context:
  \begin{itemize}
    \item read or write?
    \item instruction fetch?
    \item user access to kernel memory?
  \end{itemize}
  \item \textbf{idempotent instructions}: easy:
  \begin{itemize}
    \item re-do load/store instructions
    \item re-execute instructions accessing only one address
  \end{itemize}
  \item \textbf{Complex instructions}: must be re-started
  \begin{itemize}
    \item some CISC instructions are hard to restart (e.g., block move of overlapping areas)
    \item \emph{solutions}:
    \begin{itemize}
      \item touch relevant pages before operation starts
      \item keep modified data in registers $ \to $ page faults can't take place
      \item design ISA such that complex operations can execute partially $ \to $ consistent page fault state
    \end{itemize}
  \end{itemize}
\end{itemize}

\paragraph{Memory-Mapped Files --- other issues}
\begin{itemize}
  \item \textbf{I/O mapping}: mapping disk block to page in memory allows file I/O to be treated as routine memory
  \begin{itemize}
    \item \emph{initial}: read page-sized portion of file from file system to physical page
    \item \emph{subsequent read/write}: treated as ordinary memory access
    \item[$ \to $] \emph{simplifies} file access, file I/O through memory instead of syscalls
    \item[$ \to $] \emph{memory-file sharing}: several processes can map to same file
  \end{itemize}
\end{itemize}

\paragraph{Shared Data Segments}
\begin{itemize}
  \item \textbf{Implementation}:
  \begin{itemize}
    \item temporary, asynchronous memory-mapped files
    \item shared pages (with allocated space on backing store)
  \end{itemize}
  \item \textbf{copy on write} (COW):
  \begin{itemize}
    \item allows both parent and child process to initially share same memory pages
    \item only modified pages are copied $ \to $ more efficient process creation
  \end{itemize}
\end{itemize}

\paragraph{Page Frame Allocation --- Local vs. Global}
\begin{itemize}
  \item \textbf{Global}: all frames considered for replacement
  \begin{itemize}
    \item does not consider page ownership
    \item one process cannot get another process's frame
    \item does not protect process from a process that hogs all memory
  \end{itemize}
  \item \textbf{Local}: only frames of faulting process are considered for replacement
  \begin{itemize}
    \item isolates processes/users
    \item separately determine how many frames each process gets
  \end{itemize}
\end{itemize}

\paragraph{Fixed Allocation --- Equal vs. Proportional}
\begin{itemize}
  \item \textbf{Equal}: all processes get same amount of frames
  \item \textbf{Proportional}: allocate according to process size
  \begin{align*}
    s_i &\coloneqq \text{size of process } p_i \text{, } S \coloneqq \sum s_i \text{, } m \coloneqq \text{total number of frames } \\
    \Rightarrow a_i &\coloneqq \frac{s_i}{S}m \text{ allocation for } p_i
  \end{align*}
\end{itemize}

\paragraph{Fixed Allocation --- Priority Allocation}
\begin{itemize}
  \item[=] proportional allocation scheme using priorities rather than size
  \item \textbf{on page fault} of $ P_i $:
  \begin{itemize}
    \item select one of its frames for replacement or
    \item select frame from process with lower priority
  \end{itemize}
\end{itemize}

\paragraph{Memory Locality}
\begin{itemize}
  \item \textbf{Problem}: background storage much slower than memory
  \begin{itemize}
    \item paging extends memory size using background storage
    \item \emph{goal}: run near memory speed, not near background storage speed
  \end{itemize}
  \item \textbf{Pareto principle}: applies to working sets of processes
  \begin{itemize}
    \item 10\% of memory gets 90\% of references
    \item \emph{goal}: keep those 10\% in memory, rest on disk
    \item \emph{problem}: how to identify those 10\%?
  \end{itemize}
\end{itemize}

\paragraph{Thrashing}
\begin{itemize}
  \item \textbf{Problem}: system is busy swapping pages in and out
  \begin{itemize}
    \item each time one page is brought in, another page, whose contents will soon matter, is thrown out
    \item \emph{effect}: low CPU utilization, processes wait for pages to be fetched from disk
    \item \emph{consequence}: OS thinks that it needs higher degree of multiprogramming
  \end{itemize}
  \item \textbf{Reasons}:
  \begin{itemize}
    \item \emph{no temporal locality} of access pattern --- process doesn't follow Pareto principle
    \item \emph{too much multiprogramming}: each process fits individually, but too many for system
    \item \emph{memory too small} to hold hot memory of a single process (the 10\%)
    \item \emph{bad page replacement policy}
  \end{itemize}
\end{itemize}

\paragraph{Working-Set Model}
\begin{itemize}
  \item $ \Delta \coloneqq $ working-set window (fixed number of page references; e.g., 10000 instructions)
  \item $ \text{WSS}_i \coloneqq $ working set of process $ P_i $
  \begin{itemize}
    \item total number of pages referenced in most recent $ \Delta $ (varies in time)
    \begin{equation*}
      \Delta \begin{cases}
      \text{too small} &\Rightarrow \text{ will not encompass entire locality} \\
      \text{too large} &\Rightarrow \text{ will encompass several localities} \\
      = \infty &\Rightarrow \text{ will encompass entire program}
    \end{cases}
    \end{equation*}
  \end{itemize}
  \item $ D \coloneqq \sum \text{WSS}_i = $ total demand for frames
  \begin{itemize}
    \item $ D > m \leadsto $ \textbf{thrashing}
    \item[$ \to $] $ D > m \Rightarrow $ suspend a process
  \end{itemize}
\end{itemize}

\paragraph{Working Set --- Keeping track}
\begin{itemize}
  \item \textbf{Perfect}: replace page that is referenced furthest in the future (\emph{oracle})
  \item \textbf{Idea}: predict future from past
  \begin{itemize}
    \item record page references from past and extrapolate into future
    \item \emph{problem}: too expensive to make ordered list of all page references at runtime
  \end{itemize}
  \item \textbf{Idea}: sacrifice precision for speed
  \begin{itemize}
    \item MMU sets \emph{reference bit} in respective page table entry every time a page is referenced
    \item set timer to scan all page table entries for reference bits
  \end{itemize}
\end{itemize}

\paragraph{Page Fault Frequency --- Allocation scheme}
\begin{itemize}
  \item \textbf{Goal}: establish acceptable page fault rate
  \begin{itemize}
    \item \emph{actual rate too low} $ \to $ give frames to other process
    \item \emph{actual rate too high} $ \to $ allocate more frames to process
  \end{itemize}
\end{itemize}

\paragraph{Page Fetch Policy --- Demand Paging}
\begin{itemize}
  \item \textbf{Idea}: only transfer pages raising page faults
  \item \textbf{Advantages}:
  \begin{itemize}
    \item[+] only transfer what is needed
    \item[+] less memory needed by process $ \to $ higher multiprogramming degree possible
  \end{itemize}
  \item \textbf{Disadvantages}:
  \begin{itemize}
    \item[-] many initial page faults when task starts
    \item[-] more I/O operations $ \to $ more I/O overhead
  \end{itemize}
\end{itemize}

\paragraph{Page Fetch Policy --- Pre-paging}
\begin{itemize}
  \item \textbf{Idea}: speculatively transfer pages to RAM
  \begin{itemize}
    \item at every page fault: speculate what else should be loaded
    \item e.g., load entire text section when process starts
  \end{itemize}
  \item \textbf{Advantage}: improves disk I/O throughput
  \item \textbf{Disadvantages}:
  \begin{itemize}
    \item[-] wastes I/O bandwidth if page is never used
    \item[-] can destroy working set of other processes in case of page stealing
  \end{itemize}
\end{itemize}

\begin{summary}
  \begin{itemize}
    \item paging simulates a memory size of the size of the virtual memory
    \item when pages are filled via page faults, OS needs to answer some questions:
    \begin{itemize}
      \item what to eject?
      \item what to fetch?
      \item how to resume process?
    \end{itemize}
    \item different strategies to allocate frames and replace pages:
    \begin{itemize}
      \item local vs. global allocation
      \item fixed vs. proportional vs. priority allocation
    \end{itemize}
    \item \emph{thrashing} must be prevented by taking working sets of active processes into account
  \end{itemize}
\end{summary}
